
%!TEX root = ../dissertation.tex

% Conclusion
\section{Conclusion}
\label{sec:conclusion}
As the Internet of Things (IoT) continues to permeate nearly every aspect of our personal and professional lives, the issues of security and privacy are becoming increasingly critical. The enormous potential that IoT technologies promise is met with a proportionally complex threat landscape.

This review highlights the range of security threats that IoT systems currently face, including Sybil attacks, deceptive attacks, spoofing, Man-in-the-Middle attacks, buffer overflow attacks, and DDoS attacks. The potential for these attacks can be exacerbated by inherent vulnerabilities within IoT devices, such as limited computational power, memory capacities, and often a lack of in-built security measures. Meanwhile, advancements in defence strategies and technologies provide some level of optimism in securing IoT environments.

Similarly, the privacy threats associated with IoT are substantial, including data leakage, surveillance and eavesdropping, and inference attacks. These threats highlight the fact that seemingly innocent data could potentially be used to derive sensitive information about users. The emergence of AI and machine learning in the IoT context further compounds these challenges by introducing new vectors for potential attacks.

While the research continues to propose innovative countermeasures against these threats, a comprehensive solution remains elusive. It's imperative that the focus remains on building privacy and security into IoT systems from the ground up. There is a need for the development and application of robust standards and practices for securing and protecting IoT systems against these evolving threats.

IoT holds tremendous potential to revolutionize many facets of our lives, yet it also presents substantial challenges in terms of security and privacy. This literature review serves as a guide and reference for future research and practical approaches in the field of IoT security and privacy. It underscores the importance of continuous vigilance, proactive measures, and the development of robust interventions to safeguard IoT systems in an increasingly interconnected world.
