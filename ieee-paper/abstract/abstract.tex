%!TEX root = ../article.tex

% Abstract
\begin{abstract}
%The Internet of Things (IoT) encompasses an array of devices with the capability to connect to the Internet and exchange data. As the diversity and volume of these devices escalate, they bring along novel security and privacy vulnerabilities. With the surge in Internet connectivity and the emergence of IoT, sophisticated intrusions are increasingly infiltrating computer networks. Consequently, businesses are amplifying their research investments to enhance intrusion detection. Institutions are gravitating towards advanced methods with top-notch accuracy rates for assessment and validation. IoT's foray into sectors, such as healthcare, has seen a notable upswing, making it a focal point for tech researchers and innovators. Yet, a salient obstacle for IoT is addressing security and privacy concerns, amplified by the energy constraints and vastness of IoT devices. Addressing these concerns is crucial in the domain of computer security. This study introduces a machine learning-based intrusion detection system (ML-IDS) tailored for IoT network threat detection. The research primarily zeroes in on implementing ML-guided IDS strategies for IoT.


%In the initial methodology phase, once the CICIoT2023 dataset was integrated, the evaluation of machine learning performance was centered around three distinct angles: (i) multiclass classification, targeting the identification of 33 individual attacks; (ii) grouped classification, where the 7 broader attack categories, such as DDoS and DoS were taken into account; and (iii) binary classification, differentiating between malicious and benign traffic. For each classification type, the dataset was partitioned into training (80\%) and testing (20\%) subsets. Prior to the training phase, these subsets underwent normalization using the StandardScaler method, ensuring consistent data scales. The outcomes of the investigation, compared and contextualized with existing studies, were consolidated as integrated results. The performance of the system was marked by its high accuracy, reaching a notable 99.9\%, and an exceptional MCC value of 99.97\%.

%    In our initial methodology phase, we employed the min-max normalization approach on the CICIoT2023 dataset, mitigating potential data leakage during testing. This dataset encompasses a blend of modern intrusions and regular network activities, segmented into nine distinct attack categories. Subsequently, we implemented Principal Component Analysis (PCA) for dimensionality curtailment. For the ensuing analysis, six suggested machine learning algorithms were applied. Our outcomes, benchmarked against existing research, underwent evaluation based on criteria like validation dataset, accuracy, AUC, recall, F1 score, precision, kappa, and Matthew correlation coefficient (MCC). Our system demonstrated promising results with an accuracy peak of 99.9% and an MCC of 99.97%.



The Internet of Things (IoT) represents a diverse spectrum of devices designed to seamlessly connect to the Internet and exchange information. As the number and variety of these devices grow, new security and privacy challenges emerge. With the widespread integration of Internet connectivity and the ubiquity of IoT, advanced intrusions increasingly compromise computer networks. This trend has catalyzed an enhanced focus on research efforts aiming to improve intrusion detection mechanisms. In the pursuit of security, modern institutions are leaning towards sophisticated methods promising unparalleled accuracy in threat assessment and validation. The penetration of IoT into vital sectors, notably transportation and healthcare, underscores its importance, positioning it as a pivotal subject for technological research and innovation. However, significant challenges related to security and privacy persist in the IoT domain. These challenges, magnified by the energy constraints and extensive nature of IoT devices, warrant urgent attention in the realm of computer security. This study introduces a machine learning-based intrusion detection system (ML-IDS) meticulously crafted for pinpointing threats within IoT networks, with a primary emphasis on devising and deploying ML-powered IDS methodologies for IoT. In the initial stage of this research methodology, upon the integration of the CICIoT2023 dataset, the evaluation of machine learning's effectiveness was structured around three primary dimensions: (i) multiclass classification, aimed at distinguishing 33 specific attacks; (ii) grouped classification, focusing on the overarching 7 attack categories, including the likes of DDoS and DoS; and (iii) binary classification, discerning malicious from benign traffic. For optimal results, the dataset was methodically split into training (80\%) and testing (20\%) segments. These subsets, prior to the training process, underwent normalization using the StandardScaler method to ensure data uniformity. In comparison with existing research, the results were synthesized and presented cohesively. The system's efficacy was notably demonstrated, evidenced by an impressive accuracy rate of 99.9\% and an exemplary MCC score of 99.97\%.








\end{abstract}
