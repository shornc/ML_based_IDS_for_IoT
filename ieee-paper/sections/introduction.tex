
%!TEX root = ../dissertation.tex

% Introduction
\section{Introduction}
\label{sec:introduction}
%Your introduction goes here \ldots \\
%
%This is an example in how to cite a bibliography entry \cite{johndoe} \\
%
%This is an example in how to referring an acronym \gls{IEEE}


The Internet of Things (IoT) consists of a vast network of interconnected devices that can autonomously make decisions without human input. Technological developments in fields such as automatic identification, sensors, tracking, wireless communications, embedded computing, distributed services, and 5G networks have facilitated the integration of sophisticated devices into our daily routines through the Internet. IoT merges the concepts of the Internet and intelligent objects that possess the ability to communicate and engage. This novel approach has emerged as a prominent influencer in the Information and Communications Technology (ICT) sector for the future. Within the IoT sphere, an "object" can be virtually anything: from an individual with a medical implant to a vehicle with pressure sensors, or any device provided with an IP address for data transfer. Cisco estimated that by 2020, approximately 50 billion devices would be online. Moreover, they predicted that between 2013 and 2022, IoT would yield about \$14.4 trillion in revenue and cost savings for enterprises. While these interconnected devices—often referred to as IoT—offer significant advantages in terms of societal and business enhancements, the increased accessibility necessitates robust security measures. A primary concern for these networks is their high energy consumption due to limited battery reserves. Therefore, optimizing energy usage is vital for maintaining the Quality of Service (QoS) in the IoT framework. Potential IoT devices range from healthcare equipment and wearables to industrial machinery, smart TVs, and urban infrastructure that can be managed remotely. While many applications of IoT are fascinating, a significant percentage of the general public remains unfamiliar with the term.

Two primary factors contribute to prevalent security issues and overarching privacy concerns. First, many IoT devices have constraints related to memory, power consumption, and processing abilities. Consequently, conventional Internet security solutions, such as the Advanced Encryption Standard (AES) and RSA, may not be directly applicable to the IoT setting. For example, secure communications in devices with abundant resources like tablets and laptops can be achieved using methods like Transport Layer Security (TLS) or Internet Protocol Security (IPsec). However, these techniques may not be suitable for resource-constrained devices, leading to potential security breaches. Therefore, embedding machine learning-driven intrusion detection in the IoT model is essential to counter these vulnerabilities. The pervasive nature of IoT in our lives connects physical entities to online services, driving advancements in home automation, manufacturing, healthcare, and urban development. As we transition towards a connected, knowledge-driven society, our economies, societies, administrative apparatus, and Critical National Infrastructure (CNI) are increasingly reliant on IoT. Notable applications include smart homes, transportation, healthcare, cities, and grids. Although CNIs facilitate numerous daily operations, their dependency on IoT introduces substantial security risks. High-impact security breaches targeting CNIs, such as data leaks, denial of service attacks, and energy drainage, could lead to severe systemic failures, emphasizing the need for timely intervention. The distinct service requirements of IoT, such as resource constraints, distribution, and scalability, among others, make intrusion detection in IoT inherently different from traditional systems. Thus, it is imperative to focus on continuous research regarding intrusion detection in IoT networks. The convergence of the IPv6 protocol with IoT presents additional challenges, as it allows a plethora of devices, ranging from kitchen appliances to wearable gadgets, to access the Internet. Consequently, enhancing network security by developing IoT-specific intrusion detection techniques becomes paramount.

Extensive research has been conducted to determine the best settings for intrusion detection in IoT contexts. Some studies underscored the importance of detection as a vital step in identifying anomalies within a dataset. Others advocated for distributed deep-learning frameworks for IoT network attack detection, emphasizing the potential of artificial intelligence in cybersecurity. There have also been endeavors to develop attack detection systems for IoT applications in urban settings. While some researchers have concentrated on encryption and cryptographic methodologies, key management remains a challenge in encrypted solutions, particularly in clustered environments. Given these challenges, this study introduces a machine learning-based Intrusion Detection System (IDS) optimized for IoT networks. This approach minimizes communication overhead and negates the need for foreign keys between Cluster-Head (CH) and Cluster-Node (CN) during region transitions. The rising prominence of IoT devices, especially in comparison to smartphones, means they often hold sensitive data. This trend implies a potential increase in attack vectors and the variables associated with these attacks. One of the pressing concerns in the healthcare sector is to ensure the security of IoT devices, particularly against potential cyber-attacks targeting medical systems. Established as a pivotal solution to network security, IDS is gaining traction given the profound implications of IoT in our daily lives. Thus, it is crucial to evolve and adapt machine learning-based IDS systems capable of detecting threats in the IoT landscape.

With IoT emerging as a groundbreaking frontier in the IT domain, its security aspects are receiving heightened attention globally. The importance of IDS as a means to bolster network security is currently a focal point in IoT safety discussions. Therefore, this paper endeavors to outline machine learning-driven IDS techniques tailored for IoT attack detection. The primary contributions of this study encompass:

\begin{itemize}
    \item Utilizing the Minimum-Maximum (Min-Max) normalization method to harmonize the scale of feature values.
    \item Implementing Principal Component Analysis (PCA) for dimensionality reduction, transforming data into principal components.
    \item Creating and executing an IDS that aligns with the IoT protocol, leveraging the UNSWNB-15 dataset, which contrasts with datasets obtained from conventional networks that have inherent issues.
    \item Developing several lightweight and efficient IDS models specifically for IoT networks.
    \item Benchmarking the efficacy of the proposed models against existing methodologies.
\end{itemize}
