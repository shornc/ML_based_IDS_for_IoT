%!TEX root = ../dissertation.tex

% Literature Review
\section{Literature Review}
\label{sec:Literature Review}

Extensive work and research have been carried out recently to address safety and confidentiality issues associated with \gls{IoT}. Numerous reports and surveys have been published that shed light on various security-related challenges and issues associated with IoT. In their survey, Yang et al. addressed safety and personal problems related to low-end systems, suggesting potential solutions \cite{lin2017survey}. Other researchers have examined IoT security-related issues for networks, devices, and systems \cite{Hukkeri}.

Studies by Weber, and Gopi and Rao, focused on the security challenges of IoT in four distinct stages:
\begin{itemize}
\item The inherent constraints of IoT devices
\item The necessity for lightweight computation
\item The classification of various security threats
\item The design of access control mechanisms and overall architecture\cite{weber2015internet}\cite{gopi2018survey}.
\end{itemize}
Discussions have also been held on the security issues at different layers of IoT architecture.


Weber's research also scrutinizes security and privacy challenges and presents a security framework for IoT-based devices\cite{weber2015internet}. These devices, especially low-end ones, are equipped with various sensors and can interconnect with similar devices, enabling data transmission. However, they face significant challenges related to privacy and security. Managing extensive data securely and reliably on these machines poses a real problem. Aleisa and Renaud highlighted issues and challenges associated with IoT privacy, along with its principles, threats, and proposed solutions\cite{aleisa2016privacy}.

Tewari and Gupta discussed the layered architecture of IoT devices and the new security issues that arise\cite{tewari2020security}. They examined cross-layer heterogeneous integration problems and provided tools for research in IoT. A comparison of various studies on different aspects, including simulation tools, mechanisms, and IoT device security and privacy, was conducted by Noor and Hassan in 2019. The paper delves into the existing security mechanisms for IoT devices, which include authentication processes, security encryption protocols, trust management systems, and emerging technologies developed specifically to safeguard IoT devices\cite{hassan2019current}.

An exploration of personal and safety-related problems identified by experts in IoT devices also provides insights into how privacy in IoT differs from that in other fields. The research presented new security protocols for efficient security and privacy mechanisms\cite{bamasag2015towards}. Yet, almost all connected devices face high risks and threats of being hacked.


Protecting IoT devices from threats, both natural and human, is a significant area of focus for many researchers. While data can be protected from natural hazards, the physical damage to devices may not be restorable. Cybersecurity threats can be grouped into two categories based on their objectives. The first type involves completely disabling the target device, while in the second type, the attacker seeks unauthorized administrative access to targeted devices.

As the number of connected devices increases, so do the associated problems and challenges. To overcome these issues, new and emerging technologies, such as fog computing, artificial intelligence, and blockchain, are being integrated with IoT. The integration of blockchain technology with IoT can effectively address security and privacy-related issues in IoT devices. Sengupta et al. conducted a study about industrial IoT issues, classifying security and privacy attacks based on their destructibility and providing a blockchain-based solution\cite{sengupta2020comprehensive}. Further, studies by Wang et al. and Weber discussed blockchain technology and its features, such as access management, decentralization, asymmetric encryption, and smart contracts\cite{wang2020blockchain}\cite{weber2010internet}. Khan and Salah focused on layered architecture networking, management, and communication protocols\cite{khan2018iot}.

A study by Qian et al. examined layer-based architecture security and privacy problems for IoT\cite{qian2018towards}, proposing security mechanisms that eliminate the need for a third party to protect IoT terminal devices\cite{sultan2019iot}. The potential of blockchain technology's decentralization feature in securing remote cloud, network terminal, and devices has been discussed\cite{dorri2017blockchain}\cite{naqvi2022ontology}. Bitcoin, a rapidly growing blockchain-based technology, was also discussed\cite{sarwar2021data}\cite{akram2022triple}.

Unfortunately, IoT devices are increasingly susceptible to attacks and often cannot defend themselves\cite{minoli2018blockchain}. Additionally, managing these devices after the implementation of blockchain is challenging\cite{iqbal2021business}. A potential solution for blockchain to enhance safety is confidential transmission of data\cite{sadique2018towards}\cite{shahzad2017privacy}\cite{fan2018blockchain}.
