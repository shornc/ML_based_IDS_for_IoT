%!TEX root = ../dissertation.tex

% Types of Attacks
\section{Privacy Threat Landscape}
\label{sec:privacy_threat_landscape}
As the proliferation of Internet of Things (IoT) continues unabated, concerns surrounding privacy threats have also emerged as a critical area of focus. IoT devices, due to their inherent connectivity and data-centric operation, present numerous opportunities for privacy breaches. This section explores few of these threats reviewed by this author.

\subsection{Data Leakage}
Data leakage in the context of the Internet of Things (IoT) involves the inadvertent exposure of sensitive and private data due to vulnerabilities in IoT devices. Such data can include personal information, such as a user's lifestyle patterns, eating habits, and health information, especially when the devices are part of a smart home setup\cite{park2019security}. Data leaks can occur due to various reasons, including misconfigured software settings, weak passwords, physical theft of devices, software vulnerabilities, and insider threats.

In a research study to quantify IoT privacy leakage, it was found that by analyzing large-scale mobile network traffic data from numerous IoT devices, privacy fingerprints could be generated and used in a privacy quantification framework\cite{hui2020systematically}. This essentially means that even when data is not explicitly leaked, metadata from IoT devices can still be used to derive sensitive information about users. Moreover, the interconnected nature of IoT devices can turn these data leaks into larger security issues.

Industrial IoT also faces similar concerns. Connecting industrial equipment to the internet, while having benefits in terms of uptime and efficiency, also presents significant security issues, including data leaks\cite{semiengineering2023}.

\subsection{Surveillance and Eavesdropping}
Eavesdropping is a type of cyberattack where hackers intercept, modify, or even delete data that is transmitted between devices. These attacks can occur when cyber criminals or attackers listen in to network traffic traveling over computers, servers, mobile devices and IoT devices. This process is also referred to as network snooping or sniffing.

In a more concerning scenario, researchers have shown that even encrypted WiFi traffic from IoT devices could be susceptible to privacy leakage through an out-of-network traffic eavesdropper\cite{alyami2022wifi}. Cybercriminals could potentially take over devices like cameras and speakers to spy on individuals and businesses. This could lead to corporate eavesdropping, espionage, and other serious problems\cite{em3602023}.

Efforts are being made to prevent eavesdropping in IoT systems, as the rapid growth of IoT and its integration into various applications has heightened the necessity for privacy and security measures\cite{liao2018eavesdropping}. Despite these efforts, the potential for surveillance and eavesdropping in IoT remains a significant challenge.

\subsection{Inference Attacks}
Inference attacks in the context of the Internet of Things (IoT) represent a significant privacy and security threat. In such an attack, a data mining technique is used to analyze data and illegitimately gain knowledge about a subject or database. The attacker is able to infer sensitive information based on seemingly trivial or non-sensitive data. These attacks are particularly relevant to IoT due to the vast amount of data that these interconnected devices collect and transmit.

There are several types of inference attacks that could potentially be carried out in an IoT context. For instance, a practical membership inference attack has been explored against collaborative inference in industrial IoT\cite{chen2020practical}. In this type of attack, an adversary could determine whether a specific data point was part of the training set used for a machine learning model.

In the era of artificial IoT, practical feature inference attacks can occur in vertical federated learning during prediction processes\cite{yang2023practical}. This is a scenario where edge computing meets AI and IoT, making the security chalenges more complex.

Given the serious implications of inference attacks, efforts have been made to secure inference in IoT. These include performance bounds, algorithms, and identification of effective attacks on IoT sensor networks\cite{zhang2018approaches}. The integrity and trustworthiness of data and data analytics have become increasingly important concerns in IoT applications due to the rise in inference attacks.
